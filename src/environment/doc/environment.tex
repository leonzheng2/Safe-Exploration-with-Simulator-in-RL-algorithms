\documentclass{article}
\usepackage[utf8]{inputenc}

\title{Swimmer Environment - RLGlue Implementation}
\author{leon.zheng }
\date{May 2019}

\usepackage{natbib}
\usepackage{graphicx}
\usepackage{amsfonts}
\usepackage{amsmath}

\begin{document}

\maketitle

\section{State}

In the current implementation of the environment, the state is stored in a double array $u$ of size $2n+4$, where $n$ is the number of segments.
\begin{itemize}
    \item $A_0$, position of the swimmer head, is stored in $u_0, u_1$.
    \item $\dot{A_0}$, velocity of the swimmer head, is stored in $u_2, u_3$.
    \item $\theta_i$, the angle $i$ at the beginning of the segment $i$, is stored in $u_{3+i}$, for $i \in \{1, ..., n\}$.
    \item $\dot{\theta}_i$, the first derivative of the angle $i$ at the beginning of the segment $i$, is stored in $u_{3+n+i}$, for $i \in \{1, ..., n\}$.
\end{itemize}

\section{Solving dynamics equations}

\subsection{Variables}
\begin{itemize}
    \item $n$ is the number of segments.
    \item $l_i$ is the length of segment $i$.
    \item $\theta_i$ is the angle $i$ at the beginning of the segment $i$, for $i \in \{1, ..., n\}$. See the figure in Remy's paper, Annex B.4 \cite{Coulom-2002a}.
    \item $G_i$ is the mass center of the segment $i$, for $i \in \{1, ..., n\}$.
\end{itemize}

\subsection{Unknowns}
The goal is to compute the following unknowns:
\begin{itemize}
  \item $n$ unknowns of dimension 1: $\ddot{\theta_1}, \ddot{\theta_2}, ... , \ddot{\theta_n} \in \mathbb{R}$
  \item $n+1$ unknowns of dimension 2: $\vec{f_0}, \vec{f_1}, ..., \vec{f_n} \in \mathbb{R}^2$
  \item $n$ unknowns of dimension 2: $\ddot{G_1}, \ddot{G_2}, ..., \ddot{G_n} \in \mathbb{R}^2$
\end{itemize}

Let $X \in \mathbb{R}^{5n+2}$ be the unknown vector encoding all the unknown variables. We index the components of this vector from $0$ to $5n+1$:
\begin{itemize}
  \item $\ddot{\theta_i}$ corresponds to $X[i-1]$ for $i \in \{1, ..., n\}$.
  \item $\vec{f_i}=\begin{pmatrix} f_i^x \\ f_i^y \end{pmatrix}$ corresponds to     
    $\begin{pmatrix}
        X[n+2i] \\
        X[n+2i+1] 
    \end{pmatrix}$ for $i \in \{0, ..., n\}$.
  \item $\ddot{G_i}$ corresponds to 
    $\begin{pmatrix}
        X[3n+2i] \\
        X[3n+2i+1] 
    \end{pmatrix}$ for $i \in \{1, ..., n\}$.
\end{itemize}

\subsection{Equations}
Since $X \in \mathbb{R}^{5n+2}$, we need to write $5n+2$ equations. From Remy's model \cite{Coulom-2002a} in Annex B.4.3, we have the following equations.

\begin{itemize}
    \item Equations $0$ to $n-1$, for $i \in \{1, ..., n\}$:
    \begin{equation}
        m_i \frac{l_i}{12} \ddot{\theta_i} - \frac{l_i}{2} \left(  \cos\theta_i (f_i^y + f_{i+1}^y) - \sin\theta_i (f_i^x + f_{i+1}^x) \right)= \mathcal{M}_i - u_i + u_{i-1}
    \end{equation}
    \item Equations $n$ to $3n+3$, for $i \in \{1, ..., n\}$:
    \begin{equation}
        \vec{f_0} = 0
    \end{equation}
    \begin{equation}
        \vec{f_{i-1}} - \vec{f_i} + m_i \ddot{G_i} = \vec{F_i}
    \end{equation}
    \begin{equation}
        \vec{f_n} = 0
    \end{equation}
    
    \item Equations $3n+4$ to $5n+1$, for $i \in \{1, ..., n-1\}$:
    \begin{equation}
    \begin{split}
    -\ddot{G}_{i+1} + \ddot{G}_{i} + \frac{l_i}{2} \ddot{\theta}_{i} \begin{pmatrix}
        -\sin \theta_{i} \\
        \cos \theta_{i} 
    \end{pmatrix} + \\ \frac{l_{i+1}}{2} \ddot{\theta}_{i+1} \begin{pmatrix}
        -\sin \theta_{i+1} \\
        \cos \theta_{i+1} 
    \end{pmatrix} =  
    & \frac{l_i}{2} \dot{\theta}_{i}^2 \begin{pmatrix}
        \cos \theta_{i} \\
        \sin \theta_{i} 
    \end{pmatrix} + \frac{l_{i+1}}{2} \dot{\theta}_{i+1}^2 \begin{pmatrix}
        \cos \theta_{i+1} \\
        \sin \theta_{i+1} 
    \end{pmatrix}
    \end{split}
    \end{equation}
    
    which is the second derivative of the following equations:
    \begin{equation}
        G_{i+1} = G_i + \frac{l_i}{2} \begin{pmatrix}
        \cos \theta_{i} \\
        \sin \theta_{i} 
    \end{pmatrix} + \frac{l_i}{2} \begin{pmatrix}
        \cos \theta_{i+1} \\
        \sin \theta_{i+1} 
    \end{pmatrix}
    \end{equation}

\end{itemize}

Using the same indices of the unknowns as $X$, left members of the previous equations are stored in $A \in  \mathbb{R}^{(5n+2) \times (5n+2)}$, right members of the previous equations are stored in $B \in \mathbb{R}^{5n+2}$.

Thus, we have the relation $AX = B$, and solving the linear system is done by using $X = A^{-1} B$.

\bibliography{references}{}
\bibliographystyle{plain}
\end{document}
